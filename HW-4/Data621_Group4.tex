\documentclass[]{article}
\usepackage{lmodern}
\usepackage{amssymb,amsmath}
\usepackage{ifxetex,ifluatex}
\usepackage{fixltx2e} % provides \textsubscript
\ifnum 0\ifxetex 1\fi\ifluatex 1\fi=0 % if pdftex
  \usepackage[T1]{fontenc}
  \usepackage[utf8]{inputenc}
\else % if luatex or xelatex
  \ifxetex
    \usepackage{mathspec}
  \else
    \usepackage{fontspec}
  \fi
  \defaultfontfeatures{Ligatures=TeX,Scale=MatchLowercase}
\fi
% use upquote if available, for straight quotes in verbatim environments
\IfFileExists{upquote.sty}{\usepackage{upquote}}{}
% use microtype if available
\IfFileExists{microtype.sty}{%
\usepackage{microtype}
\UseMicrotypeSet[protrusion]{basicmath} % disable protrusion for tt fonts
}{}
\usepackage[margin=1in]{geometry}
\usepackage{hyperref}
\hypersetup{unicode=true,
            pdftitle={Multiple Linear Regression and Binary Logistic Regression},
            pdfauthor={Sachid Deshmukh; Michael Yampol; Vishal Arora; Ann Liu-Ferrara},
            pdfborder={0 0 0},
            breaklinks=true}
\urlstyle{same}  % don't use monospace font for urls
\usepackage{color}
\usepackage{fancyvrb}
\newcommand{\VerbBar}{|}
\newcommand{\VERB}{\Verb[commandchars=\\\{\}]}
\DefineVerbatimEnvironment{Highlighting}{Verbatim}{commandchars=\\\{\}}
% Add ',fontsize=\small' for more characters per line
\usepackage{framed}
\definecolor{shadecolor}{RGB}{248,248,248}
\newenvironment{Shaded}{\begin{snugshade}}{\end{snugshade}}
\newcommand{\AlertTok}[1]{\textcolor[rgb]{0.94,0.16,0.16}{#1}}
\newcommand{\AnnotationTok}[1]{\textcolor[rgb]{0.56,0.35,0.01}{\textbf{\textit{#1}}}}
\newcommand{\AttributeTok}[1]{\textcolor[rgb]{0.77,0.63,0.00}{#1}}
\newcommand{\BaseNTok}[1]{\textcolor[rgb]{0.00,0.00,0.81}{#1}}
\newcommand{\BuiltInTok}[1]{#1}
\newcommand{\CharTok}[1]{\textcolor[rgb]{0.31,0.60,0.02}{#1}}
\newcommand{\CommentTok}[1]{\textcolor[rgb]{0.56,0.35,0.01}{\textit{#1}}}
\newcommand{\CommentVarTok}[1]{\textcolor[rgb]{0.56,0.35,0.01}{\textbf{\textit{#1}}}}
\newcommand{\ConstantTok}[1]{\textcolor[rgb]{0.00,0.00,0.00}{#1}}
\newcommand{\ControlFlowTok}[1]{\textcolor[rgb]{0.13,0.29,0.53}{\textbf{#1}}}
\newcommand{\DataTypeTok}[1]{\textcolor[rgb]{0.13,0.29,0.53}{#1}}
\newcommand{\DecValTok}[1]{\textcolor[rgb]{0.00,0.00,0.81}{#1}}
\newcommand{\DocumentationTok}[1]{\textcolor[rgb]{0.56,0.35,0.01}{\textbf{\textit{#1}}}}
\newcommand{\ErrorTok}[1]{\textcolor[rgb]{0.64,0.00,0.00}{\textbf{#1}}}
\newcommand{\ExtensionTok}[1]{#1}
\newcommand{\FloatTok}[1]{\textcolor[rgb]{0.00,0.00,0.81}{#1}}
\newcommand{\FunctionTok}[1]{\textcolor[rgb]{0.00,0.00,0.00}{#1}}
\newcommand{\ImportTok}[1]{#1}
\newcommand{\InformationTok}[1]{\textcolor[rgb]{0.56,0.35,0.01}{\textbf{\textit{#1}}}}
\newcommand{\KeywordTok}[1]{\textcolor[rgb]{0.13,0.29,0.53}{\textbf{#1}}}
\newcommand{\NormalTok}[1]{#1}
\newcommand{\OperatorTok}[1]{\textcolor[rgb]{0.81,0.36,0.00}{\textbf{#1}}}
\newcommand{\OtherTok}[1]{\textcolor[rgb]{0.56,0.35,0.01}{#1}}
\newcommand{\PreprocessorTok}[1]{\textcolor[rgb]{0.56,0.35,0.01}{\textit{#1}}}
\newcommand{\RegionMarkerTok}[1]{#1}
\newcommand{\SpecialCharTok}[1]{\textcolor[rgb]{0.00,0.00,0.00}{#1}}
\newcommand{\SpecialStringTok}[1]{\textcolor[rgb]{0.31,0.60,0.02}{#1}}
\newcommand{\StringTok}[1]{\textcolor[rgb]{0.31,0.60,0.02}{#1}}
\newcommand{\VariableTok}[1]{\textcolor[rgb]{0.00,0.00,0.00}{#1}}
\newcommand{\VerbatimStringTok}[1]{\textcolor[rgb]{0.31,0.60,0.02}{#1}}
\newcommand{\WarningTok}[1]{\textcolor[rgb]{0.56,0.35,0.01}{\textbf{\textit{#1}}}}
\usepackage{graphicx,grffile}
\makeatletter
\def\maxwidth{\ifdim\Gin@nat@width>\linewidth\linewidth\else\Gin@nat@width\fi}
\def\maxheight{\ifdim\Gin@nat@height>\textheight\textheight\else\Gin@nat@height\fi}
\makeatother
% Scale images if necessary, so that they will not overflow the page
% margins by default, and it is still possible to overwrite the defaults
% using explicit options in \includegraphics[width, height, ...]{}
\setkeys{Gin}{width=\maxwidth,height=\maxheight,keepaspectratio}
\IfFileExists{parskip.sty}{%
\usepackage{parskip}
}{% else
\setlength{\parindent}{0pt}
\setlength{\parskip}{6pt plus 2pt minus 1pt}
}
\setlength{\emergencystretch}{3em}  % prevent overfull lines
\providecommand{\tightlist}{%
  \setlength{\itemsep}{0pt}\setlength{\parskip}{0pt}}
\setcounter{secnumdepth}{0}
% Redefines (sub)paragraphs to behave more like sections
\ifx\paragraph\undefined\else
\let\oldparagraph\paragraph
\renewcommand{\paragraph}[1]{\oldparagraph{#1}\mbox{}}
\fi
\ifx\subparagraph\undefined\else
\let\oldsubparagraph\subparagraph
\renewcommand{\subparagraph}[1]{\oldsubparagraph{#1}\mbox{}}
\fi

%%% Use protect on footnotes to avoid problems with footnotes in titles
\let\rmarkdownfootnote\footnote%
\def\footnote{\protect\rmarkdownfootnote}

%%% Change title format to be more compact
\usepackage{titling}

% Create subtitle command for use in maketitle
\providecommand{\subtitle}[1]{
  \posttitle{
    \begin{center}\large#1\end{center}
    }
}

\setlength{\droptitle}{-2em}

  \title{Multiple Linear Regression and Binary Logistic Regression}
    \pretitle{\vspace{\droptitle}\centering\huge}
  \posttitle{\par}
  \subtitle{Data 621: Homework 4 - Group 4}
  \author{Sachid Deshmukh \\ Michael Yampol \\ Vishal Arora \\ Ann Liu-Ferrara}
    \preauthor{\centering\large\emph}
  \postauthor{\par}
      \predate{\centering\large\emph}
  \postdate{\par}
    \date{Nov.~10, 2019}


\begin{document}
\maketitle

{
\setcounter{tocdepth}{2}
\tableofcontents
}
\newpage

\hypertarget{data-exploration}{%
\section{1. Data Exploration}\label{data-exploration}}

\hypertarget{lets-load-training-dataset-and-preview.}{%
\subsubsection{\texorpdfstring{\textbf{Let's load training dataset and
preview.}}{Let's load training dataset and preview.}}\label{lets-load-training-dataset-and-preview.}}

\begin{verbatim}
##   INDEX TARGET_FLAG TARGET_AMT KIDSDRIV AGE HOMEKIDS YOJ   INCOME PARENT1
## 1     1           0          0        0  60        0  11  $67,349      No
## 2     2           0          0        0  43        0  11  $91,449      No
## 3     4           0          0        0  35        1  10  $16,039      No
## 4     5           0          0        0  51        0  14     <NA>      No
## 5     6           0          0        0  50        0  NA $114,986      No
## 6     7           1       2946        0  34        1  12 $125,301     Yes
##   HOME_VAL MSTATUS SEX     EDUCATION           JOB TRAVTIME    CAR_USE
## 1       $0    z_No   M           PhD  Professional       14    Private
## 2 $257,252    z_No   M z_High School z_Blue Collar       22 Commercial
## 3 $124,191     Yes z_F z_High School      Clerical        5    Private
## 4 $306,251     Yes   M  <High School z_Blue Collar       32    Private
## 5 $243,925     Yes z_F           PhD        Doctor       36    Private
## 6       $0    z_No z_F     Bachelors z_Blue Collar       46 Commercial
##   BLUEBOOK TIF   CAR_TYPE RED_CAR OLDCLAIM CLM_FREQ REVOKED MVR_PTS
## 1  $14,230  11    Minivan     yes   $4,461        2      No       3
## 2  $14,940   1    Minivan     yes       $0        0      No       0
## 3   $4,010   4      z_SUV      no  $38,690        2      No       3
## 4  $15,440   7    Minivan     yes       $0        0      No       0
## 5  $18,000   1      z_SUV      no  $19,217        2     Yes       3
## 6  $17,430   1 Sports Car      no       $0        0      No       0
##   CAR_AGE          URBANICITY
## 1      18 Highly Urban/ Urban
## 2       1 Highly Urban/ Urban
## 3      10 Highly Urban/ Urban
## 4       6 Highly Urban/ Urban
## 5      17 Highly Urban/ Urban
## 6       7 Highly Urban/ Urban
\end{verbatim}

\begin{verbatim}
## [1] "Number of columns =  26"
\end{verbatim}

\begin{verbatim}
## [1] "Number of rows =  8161"
\end{verbatim}

\hypertarget{as-we-can-see-in-the-preview-training-data-has-26-columns-and-8161-rows}{%
\paragraph{As we can see in the preview, training data has 26 columns
and 8161
rows}\label{as-we-can-see-in-the-preview-training-data-has-26-columns-and-8161-rows}}

\hypertarget{lets-analyze-datatypes-of-each-column.}{%
\subsubsection{\texorpdfstring{\textbf{Let's analyze datatypes of each
column.}}{Let's analyze datatypes of each column.}}\label{lets-analyze-datatypes-of-each-column.}}

\begin{verbatim}
## 'data.frame':    8161 obs. of  26 variables:
##  $ INDEX      : int  1 2 4 5 6 7 8 11 12 13 ...
##  $ TARGET_FLAG: int  0 0 0 0 0 1 0 1 1 0 ...
##  $ TARGET_AMT : num  0 0 0 0 0 ...
##  $ KIDSDRIV   : int  0 0 0 0 0 0 0 1 0 0 ...
##  $ AGE        : int  60 43 35 51 50 34 54 37 34 50 ...
##  $ HOMEKIDS   : int  0 0 1 0 0 1 0 2 0 0 ...
##  $ YOJ        : int  11 11 10 14 NA 12 NA NA 10 7 ...
##  $ INCOME     : chr  "$67,349" "$91,449" "$16,039" NA ...
##  $ PARENT1    : chr  "No" "No" "No" "No" ...
##  $ HOME_VAL   : chr  "$0" "$257,252" "$124,191" "$306,251" ...
##  $ MSTATUS    : chr  "z_No" "z_No" "Yes" "Yes" ...
##  $ SEX        : chr  "M" "M" "z_F" "M" ...
##  $ EDUCATION  : chr  "PhD" "z_High School" "z_High School" "<High School" ...
##  $ JOB        : chr  "Professional" "z_Blue Collar" "Clerical" "z_Blue Collar" ...
##  $ TRAVTIME   : int  14 22 5 32 36 46 33 44 34 48 ...
##  $ CAR_USE    : chr  "Private" "Commercial" "Private" "Private" ...
##  $ BLUEBOOK   : chr  "$14,230" "$14,940" "$4,010" "$15,440" ...
##  $ TIF        : int  11 1 4 7 1 1 1 1 1 7 ...
##  $ CAR_TYPE   : chr  "Minivan" "Minivan" "z_SUV" "Minivan" ...
##  $ RED_CAR    : chr  "yes" "yes" "no" "yes" ...
##  $ OLDCLAIM   : chr  "$4,461" "$0" "$38,690" "$0" ...
##  $ CLM_FREQ   : int  2 0 2 0 2 0 0 1 0 0 ...
##  $ REVOKED    : chr  "No" "No" "No" "No" ...
##  $ MVR_PTS    : int  3 0 3 0 3 0 0 10 0 1 ...
##  $ CAR_AGE    : int  18 1 10 6 17 7 1 7 1 17 ...
##  $ URBANICITY : chr  "Highly Urban/ Urban" "Highly Urban/ Urban" "Highly Urban/ Urban" "Highly Urban/ Urban" ...
\end{verbatim}

\hypertarget{as-we-can-see-income-parent1-home_val-mstatus-sex-education-job-car_use-bluebook-car_type-red_car-oldclaim-revoked-and-urbanicity-variables-are-qualitative.-linear-regression-and-logistic-regression-algorithms-works-best-with-numerical-variables.-we-need-to-transform-these-variagles-making-them-quantitative-so-that-we-can-use-them-for-model-training}{%
\paragraph{As we can see Income, Parent1, Home\_val, MStatus, Sex,
Education, Job, Car\_Use, BlueBook, Car\_Type, Red\_Car, OldClaim,
Revoked and UrbaniCity variables are qualitative. Linear Regression and
Logistic regression algorithms works best with numerical variables. We
need to transform these variagles making them quantitative so that we
can use them for model
training}\label{as-we-can-see-income-parent1-home_val-mstatus-sex-education-job-car_use-bluebook-car_type-red_car-oldclaim-revoked-and-urbanicity-variables-are-qualitative.-linear-regression-and-logistic-regression-algorithms-works-best-with-numerical-variables.-we-need-to-transform-these-variagles-making-them-quantitative-so-that-we-can-use-them-for-model-training}}

\hypertarget{lets-check-if-any-variables-have-missing-values.-values-which-are-null-or-na.}{%
\subsubsection{\texorpdfstring{\textbf{Let's check if any variables have
missing values. Values which are NULL or
NA.}}{Let's check if any variables have missing values. Values which are NULL or NA.}}\label{lets-check-if-any-variables-have-missing-values.-values-which-are-null-or-na.}}

\hypertarget{check-missing-values}{%
\paragraph{\texorpdfstring{\textbf{Check missing
values}}{Check missing values}}\label{check-missing-values}}

\begin{verbatim}
## [1] "Number of columns with missing values =  6"
\end{verbatim}

\begin{verbatim}
## [1] "Names of columns with missing values =  AGE, YOJ, INCOME, HOME_VAL, JOB, CAR_AGE"
\end{verbatim}

\hypertarget{we-can-see-6-variables-namely-age-yoj-income-home_val-job-and-car_age-have-missing-values.-we-will-impute-values-for-these-variables-for-better-model-accuracy.}{%
\paragraph{We can see 6 variables namely Age, Yoj, Income, Home\_VAL,
Job and Car\_Age have missing values. We will impute values for these
variables for better model
accuracy.}\label{we-can-see-6-variables-namely-age-yoj-income-home_val-job-and-car_age-have-missing-values.-we-will-impute-values-for-these-variables-for-better-model-accuracy.}}

\hypertarget{lets-check-if-the-training-data-is-class-imbalance.-a-dataset-is-called-class-imbalance-when-there-are-very-few-obesrvations-corresponding-to-minority-class.-this-is-very-important-in-deciding-model-evaluation-metrics.}{%
\subsubsection{\texorpdfstring{\textbf{Let's check if the training data
is class imbalance. A Dataset is called class imbalance when there are
very few obesrvations corresponding to minority class. This is very
important in deciding model evaluation
metrics.}}{Let's check if the training data is class imbalance. A Dataset is called class imbalance when there are very few obesrvations corresponding to minority class. This is very important in deciding model evaluation metrics.}}\label{lets-check-if-the-training-data-is-class-imbalance.-a-dataset-is-called-class-imbalance-when-there-are-very-few-obesrvations-corresponding-to-minority-class.-this-is-very-important-in-deciding-model-evaluation-metrics.}}

\hypertarget{check-class-imbalance}{%
\paragraph{\texorpdfstring{\textbf{Check class
Imbalance}}{Check class Imbalance}}\label{check-class-imbalance}}

\includegraphics{Data621_Group4_files/figure-latex/unnamed-chunk-5-1.pdf}

\hypertarget{above-barchart-indicates-that-training-dataset-is-class-imbalance.-there-are-fewer-observation-of-car-crashes-compared-to-observations-with-no-car-crash.-this-makes-the-dataset-class-imbalance.-for-logistic-regression-we-cant-rely-on-model-metrics-like-accuracy-due-to-this.-since-the-dataset-is-class-imbalance-we-will-give-more-importance-to-precision-recall-and-roc-auc-for-evaluating-logistic-regression-model.}{%
\paragraph{Above barchart indicates that training dataset is class
imbalance. There are fewer observation of car crashes compared to
observations with no car crash. This makes the dataset class imbalance.
For logistic regression we can't rely on model metrics like Accuracy due
to this. Since the dataset is class imbalance, we will give more
importance to Precision, Recall and ROC AUC for evaluating logistic
regression
model.}\label{above-barchart-indicates-that-training-dataset-is-class-imbalance.-there-are-fewer-observation-of-car-crashes-compared-to-observations-with-no-car-crash.-this-makes-the-dataset-class-imbalance.-for-logistic-regression-we-cant-rely-on-model-metrics-like-accuracy-due-to-this.-since-the-dataset-is-class-imbalance-we-will-give-more-importance-to-precision-recall-and-roc-auc-for-evaluating-logistic-regression-model.}}

\hypertarget{lets-check-if-red-color-contributes-to-more-car-crashes}{%
\subsubsection{\texorpdfstring{\textbf{Let's check if Red color
contributes to more car
crashes}}{Let's check if Red color contributes to more car crashes}}\label{lets-check-if-red-color-contributes-to-more-car-crashes}}

\includegraphics{Data621_Group4_files/figure-latex/unnamed-chunk-6-1.pdf}

\hypertarget{above-bar-chart-doesnt-show-significant-evidence-that-red-cars-are-more-accident-prone-compare-to-non-red-cars}{%
\paragraph{Above bar chart doesn't show significant evidence that Red
cars are more accident prone compare to non red
cars}\label{above-bar-chart-doesnt-show-significant-evidence-that-red-cars-are-more-accident-prone-compare-to-non-red-cars}}

\hypertarget{lets-check-if-women-are-safe-driver-compared-to-men}{%
\subsubsection{\texorpdfstring{\textbf{Let's check if women are safe
driver compared to
men}}{Let's check if women are safe driver compared to men}}\label{lets-check-if-women-are-safe-driver-compared-to-men}}

\includegraphics{Data621_Group4_files/figure-latex/unnamed-chunk-7-1.pdf}

\hypertarget{above-bar-chart-shows-that-women-have-more-car-crashes-compared-to-men.-we-can-discard-the-fact-that-in-general-women-are-safer-driver-than-men}{%
\paragraph{Above bar chart shows that women have more car crashes
compared to men. We can discard the fact that in general women are safer
driver than
men}\label{above-bar-chart-shows-that-women-have-more-car-crashes-compared-to-men.-we-can-discard-the-fact-that-in-general-women-are-safer-driver-than-men}}

\hypertarget{data-preparation}{%
\section{2. Data Preparation}\label{data-preparation}}

\hypertarget{encode-parent1-variable.-no-0-and-yes-1}{%
\paragraph{\texorpdfstring{\textbf{Encode Parent1 variable. No = 0 and
Yes =
1}}{Encode Parent1 variable. No = 0 and Yes = 1}}\label{encode-parent1-variable.-no-0-and-yes-1}}

\begin{Shaded}
\begin{Highlighting}[]
\NormalTok{train.df}\OperatorTok{$}\NormalTok{PARENT1 =}\StringTok{ }\KeywordTok{ifelse}\NormalTok{(train.df}\OperatorTok{$}\NormalTok{PARENT1 }\OperatorTok{==}\StringTok{ 'No'}\NormalTok{, }\DecValTok{0}\NormalTok{, }\DecValTok{1}\NormalTok{)}
\NormalTok{train.df}\OperatorTok{$}\NormalTok{PARENT1 =}\StringTok{ }\KeywordTok{as.numeric}\NormalTok{(train.df}\OperatorTok{$}\NormalTok{PARENT1)}
\end{Highlighting}
\end{Shaded}

\hypertarget{encode-mstatus-variable.-no-0-and-yes-1}{%
\paragraph{\texorpdfstring{\textbf{Encode MStatus variable. No = 0 and
Yes =
1}}{Encode MStatus variable. No = 0 and Yes = 1}}\label{encode-mstatus-variable.-no-0-and-yes-1}}

\begin{Shaded}
\begin{Highlighting}[]
\NormalTok{train.df}\OperatorTok{$}\NormalTok{MSTATUS =}\StringTok{ }\KeywordTok{ifelse}\NormalTok{(train.df}\OperatorTok{$}\NormalTok{MSTATUS }\OperatorTok{==}\StringTok{ 'z_No'}\NormalTok{, }\DecValTok{0}\NormalTok{, }\DecValTok{1}\NormalTok{)}
\NormalTok{train.df}\OperatorTok{$}\NormalTok{MSTATUS =}\StringTok{ }\KeywordTok{as.numeric}\NormalTok{(train.df}\OperatorTok{$}\NormalTok{MSTATUS)}
\end{Highlighting}
\end{Shaded}

\hypertarget{encode-sex-variable.-male-0-and-female-1}{%
\paragraph{\texorpdfstring{\textbf{Encode Sex variable. Male = 0 and
Female =
1}}{Encode Sex variable. Male = 0 and Female = 1}}\label{encode-sex-variable.-male-0-and-female-1}}

\begin{Shaded}
\begin{Highlighting}[]
\NormalTok{train.df}\OperatorTok{$}\NormalTok{SEX =}\StringTok{ }\KeywordTok{ifelse}\NormalTok{(train.df}\OperatorTok{$}\NormalTok{SEX }\OperatorTok{==}\StringTok{ 'M'}\NormalTok{, }\DecValTok{0}\NormalTok{, }\DecValTok{1}\NormalTok{)}
\NormalTok{train.df}\OperatorTok{$}\NormalTok{SEX =}\StringTok{ }\KeywordTok{as.numeric}\NormalTok{(train.df}\OperatorTok{$}\NormalTok{SEX)}
\end{Highlighting}
\end{Shaded}

\hypertarget{encode-education-variable.}{%
\paragraph{\texorpdfstring{\textbf{Encode Education
variable.}}{Encode Education variable.}}\label{encode-education-variable.}}

\begin{Shaded}
\begin{Highlighting}[]
\NormalTok{train.df}\OperatorTok{$}\NormalTok{EDUCATION =}\StringTok{ }\KeywordTok{as.numeric}\NormalTok{(}\KeywordTok{factor}\NormalTok{(train.df}\OperatorTok{$}\NormalTok{EDUCATION, }\DataTypeTok{order =} \OtherTok{TRUE}\NormalTok{, }\DataTypeTok{levels =} \KeywordTok{c}\NormalTok{(}\StringTok{"<High School"}\NormalTok{, }\StringTok{"z_High School"}\NormalTok{, }\StringTok{"Bachelors"}\NormalTok{, }\StringTok{"Masters"}\NormalTok{, }\StringTok{"PhD"}\NormalTok{)))}
\end{Highlighting}
\end{Shaded}

\hypertarget{encode-job-variable.}{%
\paragraph{\texorpdfstring{\textbf{Encode Job
variable.}}{Encode Job variable.}}\label{encode-job-variable.}}

\begin{Shaded}
\begin{Highlighting}[]
\NormalTok{train.df}\OperatorTok{$}\NormalTok{JOB =}\StringTok{ }\KeywordTok{as.numeric}\NormalTok{(}\KeywordTok{factor}\NormalTok{(train.df}\OperatorTok{$}\NormalTok{JOB, }\DataTypeTok{order =} \OtherTok{TRUE}\NormalTok{, }\DataTypeTok{levels =} \KeywordTok{c}\NormalTok{(}\StringTok{"Student"}\NormalTok{, }\StringTok{"Home Maker"}\NormalTok{, }\StringTok{"z_Blue Collar"}\NormalTok{, }\StringTok{"Clerical"}\NormalTok{, }\StringTok{"Professional"}\NormalTok{, }\StringTok{'Manager'}\NormalTok{, }\StringTok{'Lawyer'}\NormalTok{, }\StringTok{'Doctor'}\NormalTok{)))}
\end{Highlighting}
\end{Shaded}

\hypertarget{encode-car_use-variable.-private-0-and-commercial-1}{%
\paragraph{\texorpdfstring{\textbf{Encode Car\_Use variable. Private = 0
and Commercial =
1}}{Encode Car\_Use variable. Private = 0 and Commercial = 1}}\label{encode-car_use-variable.-private-0-and-commercial-1}}

\begin{Shaded}
\begin{Highlighting}[]
\NormalTok{train.df}\OperatorTok{$}\NormalTok{CAR_USE =}\StringTok{ }\KeywordTok{ifelse}\NormalTok{(train.df}\OperatorTok{$}\NormalTok{CAR_USE }\OperatorTok{==}\StringTok{ "Private"}\NormalTok{, }\DecValTok{0}\NormalTok{, }\DecValTok{1}\NormalTok{)}
\NormalTok{train.df}\OperatorTok{$}\NormalTok{CAR_USE  =}\StringTok{ }\KeywordTok{as.numeric}\NormalTok{(train.df}\OperatorTok{$}\NormalTok{CAR_USE)}
\end{Highlighting}
\end{Shaded}

\hypertarget{encode-job-variable.-1}{%
\paragraph{\texorpdfstring{\textbf{Encode Job
variable.}}{Encode Job variable.}}\label{encode-job-variable.-1}}

\begin{Shaded}
\begin{Highlighting}[]
\NormalTok{train.df}\OperatorTok{$}\NormalTok{CAR_TYPE =}\StringTok{ }\KeywordTok{as.numeric}\NormalTok{(}\KeywordTok{factor}\NormalTok{(train.df}\OperatorTok{$}\NormalTok{CAR_TYPE, }\DataTypeTok{order =} \OtherTok{TRUE}\NormalTok{, }\DataTypeTok{levels =} \KeywordTok{c}\NormalTok{(}\StringTok{"Minivan"}\NormalTok{, }\StringTok{"z_SUV"}\NormalTok{, }\StringTok{"Van"}\NormalTok{, }\StringTok{"Pickup"}\NormalTok{, }\StringTok{"Panel Truck"}\NormalTok{, }\StringTok{'Sports Car'}\NormalTok{)))}
\end{Highlighting}
\end{Shaded}

\hypertarget{encode-red_car-variable.-no-0-and-yes-1}{%
\paragraph{\texorpdfstring{\textbf{Encode Red\_car variable. No = 0 and
Yes =
1}}{Encode Red\_car variable. No = 0 and Yes = 1}}\label{encode-red_car-variable.-no-0-and-yes-1}}

\begin{Shaded}
\begin{Highlighting}[]
\NormalTok{train.df}\OperatorTok{$}\NormalTok{RED_CAR =}\StringTok{ }\KeywordTok{ifelse}\NormalTok{(train.df}\OperatorTok{$}\NormalTok{RED_CAR }\OperatorTok{==}\StringTok{ "no"}\NormalTok{, }\DecValTok{0}\NormalTok{, }\DecValTok{1}\NormalTok{)}
\NormalTok{train.df}\OperatorTok{$}\NormalTok{RED_CAR  =}\StringTok{ }\KeywordTok{as.numeric}\NormalTok{(train.df}\OperatorTok{$}\NormalTok{RED_CAR)}
\end{Highlighting}
\end{Shaded}

\hypertarget{encode-revoked-variable.-no-0-and-yes-1}{%
\paragraph{\texorpdfstring{\textbf{Encode Revoked variable. No = 0 and
Yes =
1}}{Encode Revoked variable. No = 0 and Yes = 1}}\label{encode-revoked-variable.-no-0-and-yes-1}}

\begin{Shaded}
\begin{Highlighting}[]
\NormalTok{train.df}\OperatorTok{$}\NormalTok{REVOKED =}\StringTok{ }\KeywordTok{ifelse}\NormalTok{(train.df}\OperatorTok{$}\NormalTok{REVOKED }\OperatorTok{==}\StringTok{ "No"}\NormalTok{, }\DecValTok{0}\NormalTok{, }\DecValTok{1}\NormalTok{)}
\NormalTok{train.df}\OperatorTok{$}\NormalTok{REVOKED  =}\StringTok{ }\KeywordTok{as.numeric}\NormalTok{(train.df}\OperatorTok{$}\NormalTok{REVOKED)}
\end{Highlighting}
\end{Shaded}

\hypertarget{encode-urban-city-variable.-rural-0-and-urban-1}{%
\paragraph{\texorpdfstring{\textbf{Encode Urban city variable. Rural = 0
and Urban =
1}}{Encode Urban city variable. Rural = 0 and Urban = 1}}\label{encode-urban-city-variable.-rural-0-and-urban-1}}

\begin{Shaded}
\begin{Highlighting}[]
\NormalTok{train.df}\OperatorTok{$}\NormalTok{URBANICITY =}\StringTok{ }\KeywordTok{ifelse}\NormalTok{(train.df}\OperatorTok{$}\NormalTok{URBANICITY }\OperatorTok{==}\StringTok{ "z_Highly Rural/ Rural"}\NormalTok{, }\DecValTok{0}\NormalTok{, }\DecValTok{1}\NormalTok{)}
\NormalTok{train.df}\OperatorTok{$}\NormalTok{URBANICITY  =}\StringTok{ }\KeywordTok{as.numeric}\NormalTok{(train.df}\OperatorTok{$}\NormalTok{URBANICITY)}
\end{Highlighting}
\end{Shaded}

\hypertarget{convert-income-home_val-bluebook-oldclaim-variable-to-quantitative-variable}{%
\paragraph{\texorpdfstring{\textbf{Convert Income, Home\_val, BlueBook,
oldClaim Variable to quantitative
variable}}{Convert Income, Home\_val, BlueBook, oldClaim Variable to quantitative variable}}\label{convert-income-home_val-bluebook-oldclaim-variable-to-quantitative-variable}}

\hypertarget{lets-do-data-imputation-for-missing-columns}{%
\subsubsection{\texorpdfstring{\textbf{Let's do data imputation for
missing
columns}}{Let's do data imputation for missing columns}}\label{lets-do-data-imputation-for-missing-columns}}

\hypertarget{which-columns-are-mssing-and-what-is-a-missing-pattern.-lets-leverage-vim-package-to-get-this-information}{%
\paragraph{Which columns are mssing and what is a missing pattern. Let's
leverage VIM package to get this
information}\label{which-columns-are-mssing-and-what-is-a-missing-pattern.-lets-leverage-vim-package-to-get-this-information}}

\begin{verbatim}
## Warning in plot.aggr(res, ...): not enough vertical space to display
## frequencies (too many combinations)
\end{verbatim}

\includegraphics{Data621_Group4_files/figure-latex/unnamed-chunk-19-1.pdf}

\begin{verbatim}
## 
##  Variables sorted by number of missings: 
##     Variable       Count
##          JOB 0.064452886
##      CAR_AGE 0.062492342
##     HOME_VAL 0.056855777
##          YOJ 0.055630437
##       INCOME 0.054527631
##          AGE 0.000735204
##        INDEX 0.000000000
##  TARGET_FLAG 0.000000000
##   TARGET_AMT 0.000000000
##     KIDSDRIV 0.000000000
##     HOMEKIDS 0.000000000
##      PARENT1 0.000000000
##      MSTATUS 0.000000000
##          SEX 0.000000000
##    EDUCATION 0.000000000
##     TRAVTIME 0.000000000
##      CAR_USE 0.000000000
##     BLUEBOOK 0.000000000
##          TIF 0.000000000
##     CAR_TYPE 0.000000000
##      RED_CAR 0.000000000
##     OLDCLAIM 0.000000000
##     CLM_FREQ 0.000000000
##      REVOKED 0.000000000
##      MVR_PTS 0.000000000
##   URBANICITY 0.000000000
\end{verbatim}

\hypertarget{from-the-above-missing-values-pattern-we-can-see-that-most-of-the-obervation-dont-have-missing-values.-non-missing-values-are-shown-in-blue.-this-is-a-good-news-and-we-can-assert-good-quality-of-data-in-this-case.}{%
\paragraph{From the above missing values pattern we can see that most of
the obervation don't have missing values. Non missing values are shown
in blue. This is a good news and we can assert good quality of data in
this
case.}\label{from-the-above-missing-values-pattern-we-can-see-that-most-of-the-obervation-dont-have-missing-values.-non-missing-values-are-shown-in-blue.-this-is-a-good-news-and-we-can-assert-good-quality-of-data-in-this-case.}}

\hypertarget{lets-use-mice-package-to-imput-missing-values}{%
\subsubsection{\texorpdfstring{\textbf{Let's use MICE package to imput
missing
values}}{Let's use MICE package to imput missing values}}\label{lets-use-mice-package-to-imput-missing-values}}

\begin{verbatim}
## 
##  iter imp variable
##   1   1  AGE  YOJ  INCOME  HOME_VAL  JOB  CAR_AGE
##   1   2  AGE  YOJ  INCOME  HOME_VAL  JOB  CAR_AGE
##   2   1  AGE  YOJ  INCOME  HOME_VAL  JOB  CAR_AGE
##   2   2  AGE  YOJ  INCOME  HOME_VAL  JOB  CAR_AGE
##   3   1  AGE  YOJ  INCOME  HOME_VAL  JOB  CAR_AGE
##   3   2  AGE  YOJ  INCOME  HOME_VAL  JOB  CAR_AGE
##   4   1  AGE  YOJ  INCOME  HOME_VAL  JOB  CAR_AGE
##   4   2  AGE  YOJ  INCOME  HOME_VAL  JOB  CAR_AGE
##   5   1  AGE  YOJ  INCOME  HOME_VAL  JOB  CAR_AGE
##   5   2  AGE  YOJ  INCOME  HOME_VAL  JOB  CAR_AGE
##   6   1  AGE  YOJ  INCOME  HOME_VAL  JOB  CAR_AGE
##   6   2  AGE  YOJ  INCOME  HOME_VAL  JOB  CAR_AGE
##   7   1  AGE  YOJ  INCOME  HOME_VAL  JOB  CAR_AGE
##   7   2  AGE  YOJ  INCOME  HOME_VAL  JOB  CAR_AGE
##   8   1  AGE  YOJ  INCOME  HOME_VAL  JOB  CAR_AGE
##   8   2  AGE  YOJ  INCOME  HOME_VAL  JOB  CAR_AGE
##   9   1  AGE  YOJ  INCOME  HOME_VAL  JOB  CAR_AGE
##   9   2  AGE  YOJ  INCOME  HOME_VAL  JOB  CAR_AGE
##   10   1  AGE  YOJ  INCOME  HOME_VAL  JOB  CAR_AGE
##   10   2  AGE  YOJ  INCOME  HOME_VAL  JOB  CAR_AGE
\end{verbatim}

\hypertarget{the-variables-which-are-highly-correlated-carry-similar-information-and-can-affect-model-accuracy.-highly-correlated-variables-also-impacts-estimation-of-model-coefficients.-lets-figure-out-which-variables-in-the-training-datasets-are-highly-correlated-to-each-other}{%
\subsubsection{\texorpdfstring{\textbf{The variables which are highly
correlated carry similar information and can affect model accuracy.
Highly correlated variables also impacts estimation of model
coefficients. Let's figure out which variables in the training datasets
are highly correlated to each
other}}{The variables which are highly correlated carry similar information and can affect model accuracy. Highly correlated variables also impacts estimation of model coefficients. Let's figure out which variables in the training datasets are highly correlated to each other}}\label{the-variables-which-are-highly-correlated-carry-similar-information-and-can-affect-model-accuracy.-highly-correlated-variables-also-impacts-estimation-of-model-coefficients.-lets-figure-out-which-variables-in-the-training-datasets-are-highly-correlated-to-each-other}}

\includegraphics{Data621_Group4_files/figure-latex/unnamed-chunk-21-1.pdf}

\hypertarget{from-the-above-correlation-graph-we-can-see-that-target-variable-target_flag-is-highly-correlated-with-following-variables}{%
\paragraph{From the above correlation graph we can see that target
variable TARGET\_FLAG is highly correlated with following
variables}\label{from-the-above-correlation-graph-we-can-see-that-target-variable-target_flag-is-highly-correlated-with-following-variables}}

\begin{itemize}
\tightlist
\item
  \textbf{REVOKED: License Revoked}
\item
  \textbf{MVR\_PTS: Motor Vehicle Record Points}
\item
  \textbf{OLD\_CLAIM: Total Claims}
\item
  \textbf{CLAIM\_FREQ: Claim Frequency}
\item
  \textbf{URBANICITY: Home/Work Area}
\item
  \textbf{JOB: Job}
\item
  \textbf{INCOME: Income}
\item
  \textbf{HOME\_VAL: Home Value}
\item
  \textbf{MSTATUS: Maritial Status}
\item
  \textbf{CAR\_USE: Use of car}
\end{itemize}

\hypertarget{in-addition-to-above-variables-following-variables-are-important-for-target_amt-output-variable}{%
\paragraph{In addition to above variables, following variables are
important for Target\_Amt output
variable}\label{in-addition-to-above-variables-following-variables-are-important-for-target_amt-output-variable}}

\begin{itemize}
\tightlist
\item
  \textbf{BLUEBOOK: Value of vehicle}
\item
  \textbf{CAR\_AGE: Age of car}
\item
  \textbf{CAR\_TYPE: TType of car}
\end{itemize}

\hypertarget{add-interaction-terms-to-our-dataset}{%
\subsubsection{\texorpdfstring{\textbf{Add interaction terms to our
dataset}}{Add interaction terms to our dataset}}\label{add-interaction-terms-to-our-dataset}}

\begin{Shaded}
\begin{Highlighting}[]
\NormalTok{train.df}\OperatorTok{$}\NormalTok{JOB_EDU =}\StringTok{  }\KeywordTok{round}\NormalTok{(}\KeywordTok{log}\NormalTok{(train.df}\OperatorTok{$}\NormalTok{JOB }\OperatorTok{*}\StringTok{ }\NormalTok{train.df}\OperatorTok{$}\NormalTok{EDUCATION))}
\NormalTok{train.df}\OperatorTok{$}\NormalTok{HOME_INCOME =}\StringTok{ }\KeywordTok{log}\NormalTok{(}\DecValTok{1}\OperatorTok{+}\StringTok{ }\NormalTok{train.df}\OperatorTok{$}\NormalTok{HOME_VAL) }\OperatorTok{*}\StringTok{ }\KeywordTok{log}\NormalTok{(}\DecValTok{1} \OperatorTok{+}\StringTok{ }\NormalTok{train.df}\OperatorTok{$}\NormalTok{INCOME)}
\NormalTok{train.df}\OperatorTok{$}\NormalTok{MVR_PTS_Trans =}\StringTok{ }\KeywordTok{round}\NormalTok{(}\KeywordTok{log}\NormalTok{(}\DecValTok{1}\OperatorTok{+}\StringTok{ }\NormalTok{train.df}\OperatorTok{$}\NormalTok{MVR_PTS))}
\NormalTok{train.df}\OperatorTok{$}\NormalTok{AGE_SEX =}\KeywordTok{log}\NormalTok{(}\DecValTok{1} \OperatorTok{+}\StringTok{ }\NormalTok{train.df}\OperatorTok{$}\NormalTok{AGE) }\OperatorTok{*}\StringTok{ }\NormalTok{(}\DecValTok{1}\OperatorTok{+}\NormalTok{train.df}\OperatorTok{$}\NormalTok{SEX) }
\end{Highlighting}
\end{Shaded}

\hypertarget{build-models---logistic-regression}{%
\section{3. Build Models - Logistic
Regression}\label{build-models---logistic-regression}}

\hypertarget{model-fitting-and-evaluation}{%
\subsubsection{\texorpdfstring{\textbf{Model fitting and
evaluation}}{Model fitting and evaluation}}\label{model-fitting-and-evaluation}}

\hypertarget{for-model-evaluation-we-will-use-train-test-split-technique.-since-goal-of-the-exercise-is-to-predict-car-crashes-we-will-build-high-recall-model}{%
\subsubsection{\texorpdfstring{\textbf{For model evaluation we will use
train test split technique. Since goal of the exercise is to predict car
crashes, we will build high recall
model}}{For model evaluation we will use train test split technique. Since goal of the exercise is to predict car crashes, we will build high recall model}}\label{for-model-evaluation-we-will-use-train-test-split-technique.-since-goal-of-the-exercise-is-to-predict-car-crashes-we-will-build-high-recall-model}}

\hypertarget{high-recall-model}{%
\subsubsection{\texorpdfstring{\textbf{High Recall
Model}}{High Recall Model}}\label{high-recall-model}}

\hypertarget{high-recall-model-focuses-on-identifying-maximum-possible-positive-instance.-in-this-case-it-means-we-are-optimizing-our-model-to-identify-as-much-potential-car-crash-target-as-possible.-note-that-sometimes-this-can-come-at-the-cost-of-precision-where-we-might-get-high-number-of-false-positives}{%
\paragraph{High recall model focuses on identifying maximum possible
positive instance. In this case it means we are optimizing our model to
identify as much potential car crash target as possible. Note that
sometimes this can come at the cost of precision where we might get high
number of false
positives}\label{high-recall-model-focuses-on-identifying-maximum-possible-positive-instance.-in-this-case-it-means-we-are-optimizing-our-model-to-identify-as-much-potential-car-crash-target-as-possible.-note-that-sometimes-this-can-come-at-the-cost-of-precision-where-we-might-get-high-number-of-false-positives}}

\hypertarget{model-evaluation-metrics}{%
\subsubsection{\texorpdfstring{\textbf{Model Evaluation
Metrics}}{Model Evaluation Metrics}}\label{model-evaluation-metrics}}

\hypertarget{we-will-use-following-metrics-for-model-evaluation-and-comparison}{%
\paragraph{We will use following metrics for model evaluation and
comparison}\label{we-will-use-following-metrics-for-model-evaluation-and-comparison}}

\begin{itemize}
\item
  \textbf{ROC AUC} : AUC - ROC curve is a performance measurement for
  classification problem at various thresholds settings. ROC is a
  probability curve and AUC represents degree or measure of
  separability. It tells how much model is capable of distinguishing
  between classes. Higher the AUC, better the model is
\item
  \textbf{Model Accuracy} : Accuracy is one metric for evaluating
  classification models. Informally, accuracy is the fraction of
  predictions our model got right. Formally, accuracy has the following
  definition: Accuracy = (TP + TN)/(TP + FP + TN + FN)
\item
  \textbf{Model Recall} : Recall is a metrics that focuses on how many
  true positives are identified from total positive observations in the
  data set. Formally Recall has following definition: Recall = TP/TP +
  FN
\item
  \textbf{Model Precision} : Model precision is a metric that focuses on
  how many observations are truly positive out of totally identified
  positives. Formally Precision has following definition : Precision =
  TP/TP + FP
\end{itemize}

\textbf{Where}

\begin{itemize}
\tightlist
\item
  \textbf{TP} : Stands for True Positives
\item
  \textbf{TN} : Stands for True Negatives
\item
  \textbf{FP} : Stands for False Positives
\item
  \textbf{FN} : Stands for False Negatives
\end{itemize}

\hypertarget{now-we-are-clear-on-our-model-fitting-and-evaluation-method-train-test-split-and-also-have-model-evaluation-metrics-recall-which-we-will-use-to-compare-the-model-effectiveness-we-are-all-set-to-build-different-models-and-access-its-performance}{%
\subsubsection{\texorpdfstring{\textbf{Now we are clear on our model
fitting and evaluation method (Train test split) and also have model
evaluation metrics (Recall) which we will use to compare the model
effectiveness we are all set to build different models and access it's
performance}}{Now we are clear on our model fitting and evaluation method (Train test split) and also have model evaluation metrics (Recall) which we will use to compare the model effectiveness we are all set to build different models and access it's performance}}\label{now-we-are-clear-on-our-model-fitting-and-evaluation-method-train-test-split-and-also-have-model-evaluation-metrics-recall-which-we-will-use-to-compare-the-model-effectiveness-we-are-all-set-to-build-different-models-and-access-its-performance}}

\hypertarget{we-will-build-three-different-models-and-compare-them-using-above-mentioned-model-metrics}{%
\subsubsection{\texorpdfstring{\textbf{We will build three different
models and compare them using above mentioned model
metrics}}{We will build three different models and compare them using above mentioned model metrics}}\label{we-will-build-three-different-models-and-compare-them-using-above-mentioned-model-metrics}}

\hypertarget{lets-build-model-with-important-predictors-as-per-above-analysis}{%
\subsubsection{\texorpdfstring{\textbf{1. Let's build model with
important predictors as per above
analysis}}{1. Let's build model with important predictors as per above analysis}}\label{lets-build-model-with-important-predictors-as-per-above-analysis}}

\begin{itemize}
\tightlist
\item
  \textbf{REVOKED: License Revoked}
\item
  \textbf{MVR\_PTS: Motor Vehicle Record Points}
\item
  \textbf{OLD\_CLAIM: Total Claims}
\item
  \textbf{CLAIM\_FREQ: Claim Frequency}
\item
  \textbf{URBANICITY: Home/Work Area}
\item
  \textbf{JOB: Job}
\item
  \textbf{INCOME: Income}
\item
  \textbf{HOME\_VAL: Home Value}
\item
  \textbf{MSTATUS: Maritial Status}
\item
  \textbf{CAR\_USE: Use of car}
\item
  \textbf{CAR\_TYPE: Type of Car}
\end{itemize}

\hypertarget{model-with-selected-important-variables}{%
\subsubsection{\texorpdfstring{\textbf{1. Model with selected important
variables}}{1. Model with selected important variables}}\label{model-with-selected-important-variables}}

\begin{Shaded}
\begin{Highlighting}[]
\NormalTok{model.metrix =}\StringTok{ }\KeywordTok{model.fit.evaluate}\NormalTok{(}\StringTok{"target ~ INCOME + HOME_VAL + MSTATUS  + JOB + CAR_USE + CAR_TYPE + OLDCLAIM + CLM_FREQ + REVOKED + MVR_PTS + URBANICITY"}\NormalTok{, train.data, test.data)}
\KeywordTok{print.model.matrix}\NormalTok{(}\StringTok{"Base Model"}\NormalTok{, model.metrix)}
\end{Highlighting}
\end{Shaded}

\begin{verbatim}
## [1] "Printing Metrix for model:  Base Model"
## [1] "AUC : 0.781038017616685"
## [1] "Accuracy : 0.768068599428338"
## [1] "Recall : 0.287650602409639"
## [1] "Precision : 0.667832167832168"
\end{verbatim}

\hypertarget{model-with-all-the-predictor-with-transformed-variables}{%
\subsubsection{\texorpdfstring{\textbf{2. Model with all the predictor
with transformed
variables}}{2. Model with all the predictor with transformed variables}}\label{model-with-all-the-predictor-with-transformed-variables}}

\hypertarget{we-will-add-new-variables-to-model}{%
\paragraph{We will add new variables to
model}\label{we-will-add-new-variables-to-model}}

\begin{Shaded}
\begin{Highlighting}[]
\NormalTok{model.metrix =}\StringTok{ }\KeywordTok{model.fit.evaluate}\NormalTok{(}\StringTok{"target ~ KIDSDRIV + AGE + HOMEKIDS + YOJ + INCOME + PARENT1 + HOME_VAL + MSTATUS+ SEX + EDUCATION + JOB + TRAVTIME + CAR_USE + BLUEBOOK  + TIF + CAR_TYPE + RED_CAR + OLDCLAIM + CLM_FREQ + REVOKED + MVR_PTS + CAR_AGE + URBANICITY + JOB_EDU + MVR_PTS_Trans + HOME_INCOME + AGE_SEX"}\NormalTok{, train.data, test.data)}
\KeywordTok{print.model.matrix}\NormalTok{(}\StringTok{"Interaction Term"}\NormalTok{, model.metrix)}
\end{Highlighting}
\end{Shaded}

\begin{verbatim}
## [1] "Printing Metrix for model:  Interaction Term"
## [1] "AUC : 0.794767304512163"
## [1] "Accuracy : 0.775826868109432"
## [1] "Recall : 0.337349397590361"
## [1] "Precision : 0.672672672672673"
\end{verbatim}

\hypertarget{model-with-balanced-training-set}{%
\subsubsection{\texorpdfstring{\textbf{3. Model with balanced training
set}}{3. Model with balanced training set}}\label{model-with-balanced-training-set}}

\hypertarget{since-we-know-that-data-is-class-imbalance-we-will-use-rose-package-and-leverage-sythetic-data-generation-technique-to-balance-the-training-data.}{%
\paragraph{Since we know that data is class imbalance, we will use ROSE
package and leverage sythetic data generation technique to balance the
training
data.}\label{since-we-know-that-data-is-class-imbalance-we-will-use-rose-package-and-leverage-sythetic-data-generation-technique-to-balance-the-training-data.}}

\begin{Shaded}
\begin{Highlighting}[]
\NormalTok{train.data.balanced <-}\StringTok{ }\KeywordTok{ROSE}\NormalTok{(target }\OperatorTok{~}\StringTok{ }\NormalTok{., }\DataTypeTok{data =}\NormalTok{ train.data, }\DataTypeTok{seed =} \DecValTok{1}\NormalTok{)}\OperatorTok{$}\NormalTok{data}
\NormalTok{model.metrix =}\StringTok{ }\KeywordTok{model.fit.evaluate}\NormalTok{(}\StringTok{"target ~ KIDSDRIV + AGE + HOMEKIDS + YOJ + INCOME + PARENT1 + HOME_VAL + MSTATUS+ SEX + EDUCATION + JOB + TRAVTIME + CAR_USE + BLUEBOOK  + TIF + CAR_TYPE + RED_CAR + OLDCLAIM + CLM_FREQ + REVOKED + MVR_PTS + CAR_AGE + URBANICITY + JOB_EDU + MVR_PTS_Trans + HOME_INCOME + AGE_SEX"}\NormalTok{, train.data.balanced, test.data)}
\KeywordTok{print.model.matrix}\NormalTok{(}\StringTok{"Class Balanced"}\NormalTok{, model.metrix)}
\end{Highlighting}
\end{Shaded}

\begin{verbatim}
## [1] "Printing Metrix for model:  Class Balanced"
## [1] "AUC : 0.790119300732339"
## [1] "Accuracy : 0.746835443037975"
## [1] "Recall : 0.635542168674699"
## [1] "Precision : 0.5275"
\end{verbatim}

\hypertarget{from-the-above-results-we-can-see-that-model-which-uses-all-the-variables-along-with-newly-added-interaction-term-and-which-has-class-balance-data-performs-better.-we-will-use-this-model-for-prediction}{%
\paragraph{From the above results we can see that model which uses all
the variables along with newly added interaction term and which has
class balance data performs better. We will use this model for
prediction}\label{from-the-above-results-we-can-see-that-model-which-uses-all-the-variables-along-with-newly-added-interaction-term-and-which-has-class-balance-data-performs-better.-we-will-use-this-model-for-prediction}}

\hypertarget{model-coefficient-analysis}{%
\subsubsection{\texorpdfstring{\textbf{4. Model coefficient
analysis}}{4. Model coefficient analysis}}\label{model-coefficient-analysis}}

\begin{Shaded}
\begin{Highlighting}[]
\NormalTok{fit =}\StringTok{ }\KeywordTok{glm}\NormalTok{(}\DataTypeTok{formula =} \StringTok{"target ~ KIDSDRIV + AGE + HOMEKIDS + YOJ + INCOME + PARENT1 + HOME_VAL + MSTATUS+ SEX + EDUCATION + JOB + TRAVTIME + CAR_USE + BLUEBOOK  + TIF + CAR_TYPE + RED_CAR + OLDCLAIM + CLM_FREQ + REVOKED + MVR_PTS + CAR_AGE + URBANICITY + JOB_EDU + MVR_PTS_Trans + HOME_INCOME + AGE_SEX"}\NormalTok{, }\DataTypeTok{data =}\NormalTok{ train.df.class)}

\KeywordTok{summary}\NormalTok{(fit)}
\end{Highlighting}
\end{Shaded}

\begin{verbatim}
## 
## Call:
## glm(formula = "target ~ KIDSDRIV + AGE + HOMEKIDS + YOJ + INCOME + PARENT1 + HOME_VAL + MSTATUS+ SEX + EDUCATION + JOB + TRAVTIME + CAR_USE + BLUEBOOK  + TIF + CAR_TYPE + RED_CAR + OLDCLAIM + CLM_FREQ + REVOKED + MVR_PTS + CAR_AGE + URBANICITY + JOB_EDU + MVR_PTS_Trans + HOME_INCOME + AGE_SEX", 
##     data = train.df.class)
## 
## Deviance Residuals: 
##     Min       1Q   Median       3Q      Max  
## -0.9871  -0.2812  -0.1128   0.2871   1.1862  
## 
## Coefficients:
##                 Estimate Std. Error t value Pr(>|t|)    
## (Intercept)    2.192e-01  9.737e-02   2.251 0.024417 *  
## KIDSDRIV       6.135e-02  9.760e-03   6.286 3.42e-10 ***
## AGE            1.039e-03  1.501e-03   0.692 0.488812    
## HOMEKIDS       5.686e-03  5.618e-03   1.012 0.311566    
## YOJ            2.048e-04  1.284e-03   0.160 0.873257    
## INCOME        -7.131e-07  1.705e-07  -4.182 2.91e-05 ***
## PARENT1        6.852e-02  1.735e-02   3.950 7.88e-05 ***
## HOME_VAL       1.524e-07  9.602e-08   1.587 0.112449    
## MSTATUS       -6.807e-02  1.280e-02  -5.319 1.07e-07 ***
## SEX            1.856e-01  1.522e-01   1.219 0.222801    
## EDUCATION     -3.203e-02  8.504e-03  -3.766 0.000167 ***
## JOB           -1.491e-02  5.132e-03  -2.906 0.003674 ** 
## TRAVTIME       2.008e-03  2.763e-04   7.267 4.00e-13 ***
## CAR_USE        1.188e-01  1.127e-02  10.545  < 2e-16 ***
## BLUEBOOK      -4.058e-06  5.880e-07  -6.901 5.54e-12 ***
## TIF           -7.678e-03  1.044e-03  -7.356 2.07e-13 ***
## CAR_TYPE       1.722e-02  2.707e-03   6.362 2.10e-10 ***
## RED_CAR       -9.730e-03  1.278e-02  -0.762 0.446373    
## OLDCLAIM      -2.398e-06  6.373e-07  -3.762 0.000170 ***
## CLM_FREQ       3.378e-02  4.717e-03   7.162 8.65e-13 ***
## REVOKED        1.592e-01  1.486e-02  10.712  < 2e-16 ***
## MVR_PTS        3.603e-02  4.883e-03   7.378 1.77e-13 ***
## CAR_AGE       -8.086e-04  1.058e-03  -0.764 0.444658    
## URBANICITY     2.823e-01  1.178e-02  23.958  < 2e-16 ***
## JOB_EDU        2.232e-02  1.370e-02   1.629 0.103341    
## MVR_PTS_Trans -4.242e-02  1.352e-02  -3.137 0.001713 ** 
## HOME_INCOME   -6.314e-04  1.770e-04  -3.568 0.000362 ***
## AGE_SEX       -4.151e-02  3.990e-02  -1.040 0.298173    
## ---
## Signif. codes:  0 '***' 0.001 '**' 0.01 '*' 0.05 '.' 0.1 ' ' 1
## 
## (Dispersion parameter for gaussian family taken to be 0.1519754)
## 
##     Null deviance: 1585  on 8160  degrees of freedom
## Residual deviance: 1236  on 8133  degrees of freedom
## AIC: 7814.2
## 
## Number of Fisher Scoring iterations: 2
\end{verbatim}

\hypertarget{following-conclusions-can-be-drawn-from-model-coefficients}{%
\paragraph{\texorpdfstring{\textbf{Following conclusions can be drawn
from model
coefficients}}{Following conclusions can be drawn from model coefficients}}\label{following-conclusions-can-be-drawn-from-model-coefficients}}

\begin{itemize}
\tightlist
\item
  \textbf{KidsDriv: As number of kids driver increses log odds of car
  crash also increases}
\item
  \textbf{Age: Not very significant in predicting car crash}
\item
  \textbf{HomeKids: Not very significant in predicting car crash}
\item
  \textbf{YOJ: Not very significant in predicting car crash}
\item
  \textbf{Income: As Income increases log odds of car crash decreases}
\item
  \textbf{Home\_Val: Not very important for predicting car crash}
\item
  \textbf{MStatus: If you are married, it decreases the log odds of car
  crash}
\item
  \textbf{Education: As education increases log odds of car crash
  decreases}
\item
  \textbf{Jobs: Higher the job level less likely the log odds of car
  crash}
\item
  \textbf{Travtime: Longer the travel time increases the log odds of car
  crash}
\item
  \textbf{Car Use: Commercial cars have more risk compared to private
  cars}
\item
  \textbf{BlueBook: As the cost of cars increases log odds of car crash
  decreases}
\item
  \textbf{TIF: Longer the people are in force less risky it becomes}
\item
  \textbf{Red\_Car: As expected a car being red doesn't contribute to
  car crash}
\item
  \textbf{Clm\_Freq, Revoked and Mvr\_Pts: As expected all these
  variables increases the risk of car crash}
\item
  \textbf{Urbanicity : Interestigly Work area is more prone to car
  crashes than home area}
\end{itemize}

\hypertarget{from-the-above-analysis-it-is-proven-that-model-3-which-uses-all-the-variables-along-with-newly-added-interaction-term-and-which-has-class-balance-data-performs-better.-we-will-use-this-model-for-prediction.-below-bar-chart-shows-different-models-and-helps-us-to-compare-them-on-the-basis-of-model-metrics}{%
\paragraph{\texorpdfstring{\textbf{From the above analysis it is proven
that model-3 which uses all the variables along with newly added
interaction term and which has class balance data performs better. We
will use this model for prediction. Below bar chart shows different
models and helps us to compare them on the basis of model
metrics}}{From the above analysis it is proven that model-3 which uses all the variables along with newly added interaction term and which has class balance data performs better. We will use this model for prediction. Below bar chart shows different models and helps us to compare them on the basis of model metrics}}\label{from-the-above-analysis-it-is-proven-that-model-3-which-uses-all-the-variables-along-with-newly-added-interaction-term-and-which-has-class-balance-data-performs-better.-we-will-use-this-model-for-prediction.-below-bar-chart-shows-different-models-and-helps-us-to-compare-them-on-the-basis-of-model-metrics}}

\includegraphics{Data621_Group4_files/figure-latex/unnamed-chunk-29-1.pdf}

\hypertarget{build-models---multiple-linear-regression}{%
\section{4. Build Models - Multiple Linear
Regression}\label{build-models---multiple-linear-regression}}

\hypertarget{for-linear-regression-model-we-will-use-cross-validation-technique.-we-will-use-root-mean-square-as-a-model-evaluation-metric.-model-with-lower-rmse-will-be-the-best-model.}{%
\paragraph{For Linear regression model we will use cross validation
technique. We will use Root Mean Square as a model evaluation metric.
Model with lower RMSE will be the best
model.}\label{for-linear-regression-model-we-will-use-cross-validation-technique.-we-will-use-root-mean-square-as-a-model-evaluation-metric.-model-with-lower-rmse-will-be-the-best-model.}}

\hypertarget{rmse-sqrtsumactual-value---predicted-value2}{%
\paragraph{\texorpdfstring{\textbf{RMSE = sqrt(sum(Actual Value -
Predicted
Value)\^{}2)}}{RMSE = sqrt(sum(Actual Value - Predicted Value)\^{}2)}}\label{rmse-sqrtsumactual-value---predicted-value2}}

\hypertarget{model-with-selected-important-variables-1}{%
\subsubsection{\texorpdfstring{\textbf{1. Model with selected important
variables}}{1. Model with selected important variables}}\label{model-with-selected-important-variables-1}}

\begin{Shaded}
\begin{Highlighting}[]
\NormalTok{model.metrix =}\StringTok{ }\KeywordTok{lm.cv}\NormalTok{(}\StringTok{"target ~ INCOME + HOME_VAL + MSTATUS  + JOB + CAR_USE + CAR_TYPE + OLDCLAIM + CLM_FREQ + REVOKED + MVR_PTS + URBANICITY + BLUEBOOK + CAR_AGE"}\NormalTok{, train.df.reg)}
\end{Highlighting}
\end{Shaded}

\begin{verbatim}
## Warning in CVlm(data = input.data, form.lm = formula(form), m = 2, plotit = "Residual", : 
## 
##  As there is >1 explanatory variable, cross-validation
##  predicted values for a fold are not a linear function
##  of corresponding overall predicted values.  Lines that
##  are shown for the different folds are approximate
\end{verbatim}

\includegraphics{Data621_Group4_files/figure-latex/unnamed-chunk-32-1.pdf}

\begin{Shaded}
\begin{Highlighting}[]
\KeywordTok{print.rmse}\NormalTok{(model.metrix}\OperatorTok{$}\NormalTok{output)}
\end{Highlighting}
\end{Shaded}

\begin{verbatim}
## [1] "Root Mean Square =  355315.14804385"
\end{verbatim}

\hypertarget{from-the-above-residual-plot-we-can-see-that-residuals-are-y-axis-imbalance-and-heterogenious.}{%
\paragraph{From the above residual plot we can see that residuals are y
axis imbalance and
heterogenious.}\label{from-the-above-residual-plot-we-can-see-that-residuals-are-y-axis-imbalance-and-heterogenious.}}

\hypertarget{model-with-boxcox-transformation-along-with-logarithmic-transformation-of-output-variable.-we-are-also-selecting-fewer-variables-as-per-important-regression-coefficients}{%
\subsubsection{\texorpdfstring{\textbf{2. Model with BoxCox
transformation along with logarithmic transformation of output variable.
We are also selecting fewer variables as per important regression
coefficients}}{2. Model with BoxCox transformation along with logarithmic transformation of output variable. We are also selecting fewer variables as per important regression coefficients}}\label{model-with-boxcox-transformation-along-with-logarithmic-transformation-of-output-variable.-we-are-also-selecting-fewer-variables-as-per-important-regression-coefficients}}

\begin{Shaded}
\begin{Highlighting}[]
\NormalTok{bc <-}\StringTok{ }\KeywordTok{boxcox}\NormalTok{(train.df.reg}\OperatorTok{$}\NormalTok{BLUEBOOK }\OperatorTok{~}\StringTok{ }\KeywordTok{log}\NormalTok{(train.df.reg}\OperatorTok{$}\NormalTok{target))}
\end{Highlighting}
\end{Shaded}

\includegraphics{Data621_Group4_files/figure-latex/unnamed-chunk-33-1.pdf}

\begin{Shaded}
\begin{Highlighting}[]
\NormalTok{lambda <-}\StringTok{ }\NormalTok{bc}\OperatorTok{$}\NormalTok{x[}\KeywordTok{which.max}\NormalTok{(bc}\OperatorTok{$}\NormalTok{y)]}

\NormalTok{train.df.reg}\OperatorTok{$}\NormalTok{target_TRAN =}\StringTok{ }\KeywordTok{log}\NormalTok{(train.df.reg}\OperatorTok{$}\NormalTok{target)}
\NormalTok{train.df.reg}\OperatorTok{$}\NormalTok{BLUEBOOK_TRAN =}\StringTok{ }\NormalTok{(train.df.reg}\OperatorTok{$}\NormalTok{BLUEBOOK}\OperatorTok{^}\NormalTok{lambda }\DecValTok{-1}\NormalTok{)}\OperatorTok{/}\NormalTok{lambda}

\NormalTok{model.metrix =}\StringTok{ }\KeywordTok{lm.cv}\NormalTok{(}\StringTok{"target_TRAN ~ BLUEBOOK_TRAN + MVR_PTS + CAR_AGE + CAR_TYPE"}\NormalTok{, train.df.reg)}
\end{Highlighting}
\end{Shaded}

\begin{verbatim}
## Warning in CVlm(data = input.data, form.lm = formula(form), m = 2, plotit = "Residual", : 
## 
##  As there is >1 explanatory variable, cross-validation
##  predicted values for a fold are not a linear function
##  of corresponding overall predicted values.  Lines that
##  are shown for the different folds are approximate
\end{verbatim}

\includegraphics{Data621_Group4_files/figure-latex/unnamed-chunk-33-2.pdf}

\begin{Shaded}
\begin{Highlighting}[]
\NormalTok{model.metrix}\OperatorTok{$}\NormalTok{output}\OperatorTok{$}\NormalTok{Predicted =}\StringTok{ }\KeywordTok{exp}\NormalTok{(model.metrix}\OperatorTok{$}\NormalTok{output}\OperatorTok{$}\NormalTok{Predicted)}
\KeywordTok{print.rmse}\NormalTok{(model.metrix}\OperatorTok{$}\NormalTok{output)}
\end{Highlighting}
\end{Shaded}

\begin{verbatim}
## [1] "Root Mean Square =  366286.69929848"
\end{verbatim}

\hypertarget{above-residual-plot-looks-better.-it-is-balanced-on-both-the-axis-and-homogenious}{%
\paragraph{Above residual plot looks better. It is balanced on both the
axis and
homogenious}\label{above-residual-plot-looks-better.-it-is-balanced-on-both-the-axis-and-homogenious}}

\hypertarget{we-can-see-that-model-with-box-cox-transformation-along-with-target-variable-logarithmic-transformation-gives-us-higher-rmse-comapared-to-model-with-important-variable.-however-diffrence-in-rmse-is-not-that-big-and-in-general-it-is-alway-best-practice-to-select-model-with-lower-variables-toavoid-overfitting.-for-this-reason-we-will-choose-second-model-model-with-box-cox-transformation-with-fewer-variables-for-prediction}{%
\paragraph{\texorpdfstring{\textbf{We can see that model with box cox
transformation along with target variable logarithmic transformation
gives us higher RMSE comapared to model with important variable. However
diffrence in RMSE is not that big and in general it is alway best
practice to select model with lower variables toavoid overfitting. For
this reason we will choose second model (Model with box cox
transformation with fewer variables) for
prediction}}{We can see that model with box cox transformation along with target variable logarithmic transformation gives us higher RMSE comapared to model with important variable. However diffrence in RMSE is not that big and in general it is alway best practice to select model with lower variables toavoid overfitting. For this reason we will choose second model (Model with box cox transformation with fewer variables) for prediction}}\label{we-can-see-that-model-with-box-cox-transformation-along-with-target-variable-logarithmic-transformation-gives-us-higher-rmse-comapared-to-model-with-important-variable.-however-diffrence-in-rmse-is-not-that-big-and-in-general-it-is-alway-best-practice-to-select-model-with-lower-variables-toavoid-overfitting.-for-this-reason-we-will-choose-second-model-model-with-box-cox-transformation-with-fewer-variables-for-prediction}}

\hypertarget{regression-model-coefficient-analysis}{%
\subsubsection{\texorpdfstring{\textbf{3. Regression model coefficient
analysis}}{3. Regression model coefficient analysis}}\label{regression-model-coefficient-analysis}}

\begin{Shaded}
\begin{Highlighting}[]
\NormalTok{fit =}\StringTok{ }\KeywordTok{lm}\NormalTok{(target_TRAN }\OperatorTok{~}\StringTok{ }\NormalTok{BLUEBOOK_TRAN }\OperatorTok{+}\StringTok{ }\NormalTok{MVR_PTS }\OperatorTok{+}\StringTok{ }\NormalTok{CAR_AGE }\OperatorTok{+}\StringTok{ }\NormalTok{CAR_TYPE }\OperatorTok{+}\StringTok{ }\NormalTok{REVOKED }\OperatorTok{+}\StringTok{ }\NormalTok{SEX, train.df.reg)}
\KeywordTok{summary}\NormalTok{(fit)}
\end{Highlighting}
\end{Shaded}

\begin{verbatim}
## 
## Call:
## lm(formula = target_TRAN ~ BLUEBOOK_TRAN + MVR_PTS + CAR_AGE + 
##     CAR_TYPE + REVOKED + SEX, data = train.df.reg)
## 
## Residuals:
##     Min      1Q  Median      3Q     Max 
## -4.7888 -0.4018  0.0342  0.4030  3.1519 
## 
## Coefficients:
##                 Estimate Std. Error t value Pr(>|t|)    
## (Intercept)    7.865e+00  8.550e-02  91.983  < 2e-16 ***
## BLUEBOOK_TRAN  4.232e-03  7.686e-04   5.506  4.1e-08 ***
## MVR_PTS        1.395e-02  6.753e-03   2.066   0.0389 *  
## CAR_AGE        3.301e-05  3.216e-03   0.010   0.9918    
## CAR_TYPE       2.532e-03  1.054e-02   0.240   0.8102    
## REVOKED       -2.926e-02  4.300e-02  -0.681   0.4962    
## SEX           -5.424e-02  3.511e-02  -1.545   0.1226    
## ---
## Signif. codes:  0 '***' 0.001 '**' 0.01 '*' 0.05 '.' 0.1 ' ' 1
## 
## Residual standard error: 0.8062 on 2146 degrees of freedom
## Multiple R-squared:  0.01838,    Adjusted R-squared:  0.01564 
## F-statistic: 6.699 on 6 and 2146 DF,  p-value: 4.857e-07
\end{verbatim}

\hypertarget{interestingly-we-can-see-that-only-two-variables-are-statistically-significant-and-contributing-towards-target-amount-prediction.}{%
\paragraph{Interestingly we can see that only two variables are
statistically significant and contributing towards target amount
prediction.}\label{interestingly-we-can-see-that-only-two-variables-are-statistically-significant-and-contributing-towards-target-amount-prediction.}}

\begin{itemize}
\tightlist
\item
  \textbf{BLUEBOOK : Vlaue of car is very important factor in
  determining claim amount. This perfectly makes sense}
\item
  \textbf{MVR\_PTS: Number of traffic tickets is next important
  variable. Note that coefficient is positive which indicates pay out
  amount increases with number of ticket violation. This is counter
  intuitive. Ideally more the traffic violations lesser should be the
  payout. However we will use this model since this is giving us lower
  RMSE indicating better model fit}
\end{itemize}

\hypertarget{regression-model-redicual-analysis}{%
\subsubsection{\texorpdfstring{\textbf{3. Regression model redicual
analysis}}{3. Regression model redicual analysis}}\label{regression-model-redicual-analysis}}

\begin{Shaded}
\begin{Highlighting}[]
\KeywordTok{qqnorm}\NormalTok{(fit}\OperatorTok{$}\NormalTok{residuals)}
\end{Highlighting}
\end{Shaded}

\includegraphics{Data621_Group4_files/figure-latex/unnamed-chunk-35-1.pdf}

\hypertarget{from-the-residual-normality-plots-we-can-see-that-residuals-are-normally-distributed.-this-satisfies-the-normality-assumption-of-residuals-for-multiple-regression-model.}{%
\paragraph{From the residual normality plots we can see that residuals
are normally distributed. This satisfies the normality assumption of
residuals for multiple regression
model.}\label{from-the-residual-normality-plots-we-can-see-that-residuals-are-normally-distributed.-this-satisfies-the-normality-assumption-of-residuals-for-multiple-regression-model.}}

\hypertarget{select-models}{%
\section{5. Select Models}\label{select-models}}

\hypertarget{read-evaluation-data-and-clean.-create-required-interaction-terms}{%
\subsubsection{\texorpdfstring{\textbf{1. Read evaluation data and
clean. Create required interaction
terms}}{1. Read evaluation data and clean. Create required interaction terms}}\label{read-evaluation-data-and-clean.-create-required-interaction-terms}}

\begin{verbatim}
## 
##  iter imp variable
##   1   1  AGE  YOJ  INCOME  HOME_VAL  JOB  CAR_AGE
##   1   2  AGE  YOJ  INCOME  HOME_VAL  JOB  CAR_AGE
##   2   1  AGE  YOJ  INCOME  HOME_VAL  JOB  CAR_AGE
##   2   2  AGE  YOJ  INCOME  HOME_VAL  JOB  CAR_AGE
##   3   1  AGE  YOJ  INCOME  HOME_VAL  JOB  CAR_AGE
##   3   2  AGE  YOJ  INCOME  HOME_VAL  JOB  CAR_AGE
##   4   1  AGE  YOJ  INCOME  HOME_VAL  JOB  CAR_AGE
##   4   2  AGE  YOJ  INCOME  HOME_VAL  JOB  CAR_AGE
##   5   1  AGE  YOJ  INCOME  HOME_VAL  JOB  CAR_AGE
##   5   2  AGE  YOJ  INCOME  HOME_VAL  JOB  CAR_AGE
##   6   1  AGE  YOJ  INCOME  HOME_VAL  JOB  CAR_AGE
##   6   2  AGE  YOJ  INCOME  HOME_VAL  JOB  CAR_AGE
##   7   1  AGE  YOJ  INCOME  HOME_VAL  JOB  CAR_AGE
##   7   2  AGE  YOJ  INCOME  HOME_VAL  JOB  CAR_AGE
##   8   1  AGE  YOJ  INCOME  HOME_VAL  JOB  CAR_AGE
##   8   2  AGE  YOJ  INCOME  HOME_VAL  JOB  CAR_AGE
##   9   1  AGE  YOJ  INCOME  HOME_VAL  JOB  CAR_AGE
##   9   2  AGE  YOJ  INCOME  HOME_VAL  JOB  CAR_AGE
##   10   1  AGE  YOJ  INCOME  HOME_VAL  JOB  CAR_AGE
##   10   2  AGE  YOJ  INCOME  HOME_VAL  JOB  CAR_AGE
\end{verbatim}

\begin{verbatim}
## Warning: Number of logged events: 2
\end{verbatim}

\hypertarget{logistic-regression}{%
\subsubsection{\texorpdfstring{\textbf{Logistic
Regression}}{Logistic Regression}}\label{logistic-regression}}

\hypertarget{for-logistic-regression-we-will-use-model-3.-model-3-uses-balanced-dataset-using-synthetic-data-genration-technique-smote-and-also-leverages-other-variable-to-determine-risk-of-car-crashes}{%
\paragraph{For Logistic regression we will use Model-3. Model-3 uses
balanced dataset using Synthetic Data Genration technique (SMOTE) and
also leverages other variable to determine risk of car
crashes}\label{for-logistic-regression-we-will-use-model-3.-model-3-uses-balanced-dataset-using-synthetic-data-genration-technique-smote-and-also-leverages-other-variable-to-determine-risk-of-car-crashes}}

\begin{Shaded}
\begin{Highlighting}[]
\NormalTok{model.formula =}\StringTok{ "target ~ KIDSDRIV + AGE + HOMEKIDS + YOJ + INCOME + PARENT1 + HOME_VAL + MSTATUS+ SEX + EDUCATION + JOB + TRAVTIME + CAR_USE + BLUEBOOK  + TIF + CAR_TYPE + RED_CAR + OLDCLAIM + CLM_FREQ + REVOKED + MVR_PTS + CAR_AGE + URBANICITY + JOB_EDU + MVR_PTS_Trans + HOME_INCOME + AGE_SEX"}


\NormalTok{model <-}\StringTok{ }\KeywordTok{glm}\NormalTok{(}\DataTypeTok{formula =}\NormalTok{ model.formula, }\DataTypeTok{family =} \StringTok{"binomial"}\NormalTok{, }\DataTypeTok{data =}\NormalTok{ train.data.balanced)}



\NormalTok{predicted <-}\StringTok{ }\KeywordTok{predict}\NormalTok{(model, }\DataTypeTok{newdata =}\NormalTok{ testing,}\DataTypeTok{type=}\StringTok{"response"}\NormalTok{)}
\NormalTok{lables =}\StringTok{ }\KeywordTok{ifelse}\NormalTok{(predicted }\OperatorTok{>}\StringTok{ }\FloatTok{0.5}\NormalTok{, }\DecValTok{1}\NormalTok{, }\DecValTok{0}\NormalTok{)}

\NormalTok{testing}\OperatorTok{$}\NormalTok{TARGET_FLAG_Proba =}\StringTok{ }\KeywordTok{round}\NormalTok{(predicted,}\DecValTok{1}\NormalTok{)}
\NormalTok{testing}\OperatorTok{$}\NormalTok{TARGET_FLAG =}\StringTok{ }\NormalTok{lables}
\NormalTok{testing =}\StringTok{ }\KeywordTok{data.frame}\NormalTok{(testing)}
\CommentTok{#write.table(testing, "./Data/PredictedOutcome.csv", row.names = FALSE, sep=",")}
\end{Highlighting}
\end{Shaded}

\hypertarget{multiple-linear-regression}{%
\subsubsection{\texorpdfstring{\textbf{Multiple Linear
Regression}}{Multiple Linear Regression}}\label{multiple-linear-regression}}

\hypertarget{for-multiple-linear-regression-we-will-use-the-second-model.-second-model-uses-box-cox-transfomration-and-fewer-variable.-even-though-rmse-is-lower-we-are-sticking-to-this-model-due-to-fewer-variables-and-explainability}{%
\paragraph{For multiple linear regression we will use the second model.
Second model uses Box Cox transfomration and fewer variable. Even though
RMSE is lower we are sticking to this model due to fewer variables and
explainability}\label{for-multiple-linear-regression-we-will-use-the-second-model.-second-model-uses-box-cox-transfomration-and-fewer-variable.-even-though-rmse-is-lower-we-are-sticking-to-this-model-due-to-fewer-variables-and-explainability}}

\begin{Shaded}
\begin{Highlighting}[]
\NormalTok{model.formula =}\StringTok{ "target_TRAN ~ BLUEBOOK_TRAN + MVR_PTS + CAR_AGE + CAR_TYPE"}


\NormalTok{model <-}\StringTok{ }\KeywordTok{lm}\NormalTok{(}\DataTypeTok{formula =}\NormalTok{ model.formula,  }\DataTypeTok{data =}\NormalTok{ train.df.reg)}



\NormalTok{predicted <-}\StringTok{ }\KeywordTok{predict}\NormalTok{(model, }\DataTypeTok{newdata =}\NormalTok{ testing)}
\NormalTok{testing}\OperatorTok{$}\NormalTok{TARGET_AMT =}\StringTok{ }\NormalTok{testing}\OperatorTok{$}\NormalTok{TARGET_FLAG }\OperatorTok{*}\StringTok{ }\KeywordTok{exp}\NormalTok{(predicted)}
\NormalTok{testing =}\StringTok{ }\KeywordTok{data.frame}\NormalTok{(testing)}
\KeywordTok{write.table}\NormalTok{(testing, }\StringTok{"./Data/PredictedOutcome.csv"}\NormalTok{, }\DataTypeTok{row.names =} \OtherTok{FALSE}\NormalTok{, }\DataTypeTok{sep=}\StringTok{","}\NormalTok{)}
\end{Highlighting}
\end{Shaded}

\hypertarget{appendix}{%
\section{6. Appendix}\label{appendix}}

\hypertarget{train-test-split-validation-code}{%
\paragraph{\texorpdfstring{\textbf{Train Test Split Validation
Code}}{Train Test Split Validation Code}}\label{train-test-split-validation-code}}

\begin{Shaded}
\begin{Highlighting}[]
\NormalTok{model.fit.evaluate <-}\StringTok{ }\ControlFlowTok{function}\NormalTok{(model.formula, train.data, test.data) \{}
\NormalTok{  auclist =}\StringTok{ }\OtherTok{NULL}
\NormalTok{  accuracylist =}\StringTok{ }\OtherTok{NULL}
\NormalTok{  recalllist =}\StringTok{ }\OtherTok{NULL}
\NormalTok{  precisionlist =}\StringTok{ }\OtherTok{NULL}
\NormalTok{  k =}\DecValTok{1}
  \KeywordTok{set.seed}\NormalTok{(}\DecValTok{123}\NormalTok{)}
  
\NormalTok{  training <-train.data}
\NormalTok{  testing.org<-test.data}
  
\NormalTok{  testing =}\StringTok{ }\NormalTok{testing.org[}\DecValTok{1}\OperatorTok{:}\KeywordTok{nrow}\NormalTok{(testing.org), }\KeywordTok{names}\NormalTok{(testing.org)[}\KeywordTok{names}\NormalTok{(testing.org) }\OperatorTok{!=}\StringTok{ 'target'}\NormalTok{]]}
  
\NormalTok{  model <-}\StringTok{ }\KeywordTok{glm}\NormalTok{(}\DataTypeTok{formula =}\NormalTok{ model.formula,}
               \DataTypeTok{family =} \StringTok{"binomial"}\NormalTok{, }\DataTypeTok{data =}\NormalTok{ training)}
\NormalTok{  predicted <-}\StringTok{ }\KeywordTok{predict}\NormalTok{(model, }\DataTypeTok{newdata =}\NormalTok{ testing,}\DataTypeTok{type=}\StringTok{"response"}\NormalTok{)}
\NormalTok{  pred <-}\StringTok{ }\KeywordTok{prediction}\NormalTok{(predicted, testing.org}\OperatorTok{$}\NormalTok{target)}
\NormalTok{  ind =}\StringTok{ }\KeywordTok{which.max}\NormalTok{(}\KeywordTok{round}\NormalTok{(}\KeywordTok{slot}\NormalTok{(pred, }\StringTok{'cutoffs'}\NormalTok{)[[}\DecValTok{1}\NormalTok{]],}\DecValTok{1}\NormalTok{) }\OperatorTok{==}\StringTok{ }\FloatTok{0.5}\NormalTok{)}
\NormalTok{  perf <-}\StringTok{ }\KeywordTok{performance}\NormalTok{(pred, }\DataTypeTok{measure =} \StringTok{"tpr"}\NormalTok{, }\DataTypeTok{x.measure =} \StringTok{"fpr"}\NormalTok{)}
  
\NormalTok{  auc.tmp <-}\StringTok{ }\KeywordTok{performance}\NormalTok{(pred,}\StringTok{"auc"}\NormalTok{);}
\NormalTok{  auc <-}\StringTok{ }\KeywordTok{as.numeric}\NormalTok{(auc.tmp}\OperatorTok{@}\NormalTok{y.values)}
\NormalTok{  auclist[k] =}\StringTok{ }\NormalTok{auc}
  
\NormalTok{  acc.perf =}\StringTok{ }\KeywordTok{performance}\NormalTok{(pred, }\DataTypeTok{measure =} \StringTok{"acc"}\NormalTok{)}
\NormalTok{  acc =}\StringTok{ }\KeywordTok{slot}\NormalTok{(acc.perf, }\StringTok{"y.values"}\NormalTok{)[[}\DecValTok{1}\NormalTok{]][ind]}
\NormalTok{  accuracylist[k] =}\StringTok{ }\NormalTok{acc}
  
\NormalTok{  prec.perf =}\StringTok{ }\KeywordTok{performance}\NormalTok{(pred, }\DataTypeTok{measure =} \StringTok{"prec"}\NormalTok{)}
\NormalTok{  prec =}\StringTok{ }\KeywordTok{slot}\NormalTok{(prec.perf, }\StringTok{"y.values"}\NormalTok{)[[}\DecValTok{1}\NormalTok{]][ind]}
\NormalTok{  precisionlist[k] =}\StringTok{ }\NormalTok{prec}
  
\NormalTok{  recall.perf =}\StringTok{ }\KeywordTok{performance}\NormalTok{(pred, }\DataTypeTok{measure =} \StringTok{"tpr"}\NormalTok{)}
\NormalTok{  recall =}\StringTok{ }\KeywordTok{slot}\NormalTok{(recall.perf, }\StringTok{"y.values"}\NormalTok{)[[}\DecValTok{1}\NormalTok{]][ind]}
\NormalTok{  recalllist[k] =}\StringTok{ }\NormalTok{recall}
  
    
    
  \KeywordTok{return}\NormalTok{(}\KeywordTok{list}\NormalTok{(}\StringTok{"AUC"}\NormalTok{ =}\StringTok{ }\KeywordTok{mean}\NormalTok{(auclist), }\StringTok{"Accuracy"}\NormalTok{ =}\StringTok{ }\KeywordTok{mean}\NormalTok{(accuracylist),  }\StringTok{"Recall"}\NormalTok{ =}\StringTok{ }\KeywordTok{mean}\NormalTok{(recalllist), }\StringTok{"Precision"}\NormalTok{ =}\StringTok{ }\KeywordTok{mean}\NormalTok{(precisionlist)))}
\NormalTok{\}}

\NormalTok{df.metrix <<-}\StringTok{ }\OtherTok{NULL}
\NormalTok{print.model.matrix =}\StringTok{ }\ControlFlowTok{function}\NormalTok{(model.name, matrixobj)}
\NormalTok{\{}
  
  \KeywordTok{print}\NormalTok{(}\KeywordTok{paste}\NormalTok{(}\StringTok{"Printing Metrix for model: "}\NormalTok{, model.name))}
  \ControlFlowTok{for}\NormalTok{(i }\ControlFlowTok{in} \DecValTok{1} \OperatorTok{:}\StringTok{ }\KeywordTok{length}\NormalTok{(matrixobj))}
\NormalTok{  \{}
\NormalTok{    df =}\StringTok{ }\KeywordTok{data.frame}\NormalTok{(}\StringTok{"Model"}\NormalTok{ =}\StringTok{ }\NormalTok{model.name, }\StringTok{"Metrix"}\NormalTok{=}\KeywordTok{names}\NormalTok{(matrixobj)[[i]], }\StringTok{"Value"}\NormalTok{ =}\StringTok{ }\NormalTok{matrixobj[[i]])}
\NormalTok{    df.metrix <<-}\StringTok{ }\KeywordTok{rbind}\NormalTok{(df, df.metrix)}
    \KeywordTok{print}\NormalTok{(}\KeywordTok{paste}\NormalTok{(}\KeywordTok{names}\NormalTok{(matrixobj)[[i]], }\StringTok{":"}\NormalTok{, matrixobj[[i]]))}
\NormalTok{  \}}
  
\NormalTok{\}}
\end{Highlighting}
\end{Shaded}

\hypertarget{linear-regression-cross-validation-code}{%
\paragraph{\texorpdfstring{\textbf{Linear Regression Cross Validation
Code}}{Linear Regression Cross Validation Code}}\label{linear-regression-cross-validation-code}}

\begin{Shaded}
\begin{Highlighting}[]
\NormalTok{lm.cv =}\StringTok{ }\ControlFlowTok{function}\NormalTok{(form,input.data)}
\NormalTok{\{}
\NormalTok{  out <-}\StringTok{ }\KeywordTok{CVlm}\NormalTok{(}\DataTypeTok{data =}\NormalTok{ input.data, }\DataTypeTok{form.lm =} \KeywordTok{formula}\NormalTok{(form),}\DataTypeTok{m=}\DecValTok{2}\NormalTok{,}\DataTypeTok{plotit=} \StringTok{"Residual"}\NormalTok{, }\DataTypeTok{printit =} \OtherTok{FALSE}\NormalTok{)}
\NormalTok{  cv.rmse <-}\StringTok{ }\KeywordTok{sqrt}\NormalTok{(}\KeywordTok{attr}\NormalTok{(out,}\StringTok{"ms"}\NormalTok{))}
  \KeywordTok{return}\NormalTok{(}\KeywordTok{list}\NormalTok{(}\StringTok{'output'}\NormalTok{ =}\StringTok{ }\NormalTok{out))}
\NormalTok{\}}
\NormalTok{print.rmse =}\StringTok{ }\ControlFlowTok{function}\NormalTok{(output.df)}
\NormalTok{\{}
\NormalTok{  rss =}\StringTok{ }\KeywordTok{sum}\NormalTok{((output.df}\OperatorTok{$}\NormalTok{target }\OperatorTok{-}\StringTok{ }\NormalTok{output.df}\OperatorTok{$}\NormalTok{Predicted)}\OperatorTok{^}\DecValTok{2}\NormalTok{)}
  \KeywordTok{print}\NormalTok{(}\KeywordTok{paste}\NormalTok{(}\StringTok{'Root Mean Square = '}\NormalTok{, }\KeywordTok{sqrt}\NormalTok{(rss)))}
\NormalTok{\}}
\end{Highlighting}
\end{Shaded}

\hypertarget{evaluation-data-cleanup-code}{%
\paragraph{\texorpdfstring{\textbf{Evaluation data cleanup
code}}{Evaluation data cleanup code}}\label{evaluation-data-cleanup-code}}

\begin{Shaded}
\begin{Highlighting}[]
\NormalTok{testing =}\StringTok{ }\KeywordTok{read.csv}\NormalTok{(}\StringTok{"./Data/insurance-evaluation-data.csv"}\NormalTok{, }\DataTypeTok{stringsAsFactors =} \OtherTok{FALSE}\NormalTok{, }\DataTypeTok{na.strings=}\KeywordTok{c}\NormalTok{(}\StringTok{"NA"}\NormalTok{,}\StringTok{"NaN"}\NormalTok{, }\StringTok{" "}\NormalTok{, }\StringTok{""}\NormalTok{))}

\NormalTok{testing}\OperatorTok{$}\NormalTok{PARENT1 =}\StringTok{ }\KeywordTok{ifelse}\NormalTok{(testing}\OperatorTok{$}\NormalTok{PARENT1 }\OperatorTok{==}\StringTok{ 'No'}\NormalTok{, }\DecValTok{0}\NormalTok{, }\DecValTok{1}\NormalTok{)}
\NormalTok{testing}\OperatorTok{$}\NormalTok{PARENT1 =}\StringTok{ }\KeywordTok{as.numeric}\NormalTok{(testing}\OperatorTok{$}\NormalTok{PARENT1)}
\NormalTok{testing}\OperatorTok{$}\NormalTok{MSTATUS =}\StringTok{ }\KeywordTok{ifelse}\NormalTok{(testing}\OperatorTok{$}\NormalTok{MSTATUS }\OperatorTok{==}\StringTok{ 'z_No'}\NormalTok{, }\DecValTok{0}\NormalTok{, }\DecValTok{1}\NormalTok{)}
\NormalTok{testing}\OperatorTok{$}\NormalTok{MSTATUS =}\StringTok{ }\KeywordTok{as.numeric}\NormalTok{(testing}\OperatorTok{$}\NormalTok{MSTATUS)}
\NormalTok{testing}\OperatorTok{$}\NormalTok{SEX =}\StringTok{ }\KeywordTok{ifelse}\NormalTok{(testing}\OperatorTok{$}\NormalTok{SEX }\OperatorTok{==}\StringTok{ 'M'}\NormalTok{, }\DecValTok{0}\NormalTok{, }\DecValTok{1}\NormalTok{)}
\NormalTok{testing}\OperatorTok{$}\NormalTok{SEX =}\StringTok{ }\KeywordTok{as.numeric}\NormalTok{(testing}\OperatorTok{$}\NormalTok{SEX)}
\NormalTok{testing}\OperatorTok{$}\NormalTok{EDUCATION =}\StringTok{ }\KeywordTok{as.numeric}\NormalTok{(}\KeywordTok{factor}\NormalTok{(testing}\OperatorTok{$}\NormalTok{EDUCATION, }\DataTypeTok{order =} \OtherTok{TRUE}\NormalTok{, }\DataTypeTok{levels =} \KeywordTok{c}\NormalTok{(}\StringTok{"<High School"}\NormalTok{, }\StringTok{"z_High School"}\NormalTok{, }\StringTok{"Bachelors"}\NormalTok{, }\StringTok{"Masters"}\NormalTok{, }\StringTok{"PhD"}\NormalTok{)))}
\NormalTok{testing}\OperatorTok{$}\NormalTok{JOB =}\StringTok{ }\KeywordTok{as.numeric}\NormalTok{(}\KeywordTok{factor}\NormalTok{(testing}\OperatorTok{$}\NormalTok{JOB, }\DataTypeTok{order =} \OtherTok{TRUE}\NormalTok{, }\DataTypeTok{levels =} \KeywordTok{c}\NormalTok{(}\StringTok{"Student"}\NormalTok{, }\StringTok{"Home Maker"}\NormalTok{, }\StringTok{"z_Blue Collar"}\NormalTok{, }\StringTok{"Clerical"}\NormalTok{, }\StringTok{"Professional"}\NormalTok{, }\StringTok{'Manager'}\NormalTok{, }\StringTok{'Lawyer'}\NormalTok{, }\StringTok{'Doctor'}\NormalTok{)))}
\NormalTok{testing}\OperatorTok{$}\NormalTok{CAR_USE =}\StringTok{ }\KeywordTok{ifelse}\NormalTok{(testing}\OperatorTok{$}\NormalTok{CAR_USE }\OperatorTok{==}\StringTok{ "Private"}\NormalTok{, }\DecValTok{0}\NormalTok{, }\DecValTok{1}\NormalTok{)}
\NormalTok{testing}\OperatorTok{$}\NormalTok{CAR_USE  =}\StringTok{ }\KeywordTok{as.numeric}\NormalTok{(testing}\OperatorTok{$}\NormalTok{CAR_USE)}
\NormalTok{testing}\OperatorTok{$}\NormalTok{CAR_TYPE =}\StringTok{ }\KeywordTok{as.numeric}\NormalTok{(}\KeywordTok{factor}\NormalTok{(testing}\OperatorTok{$}\NormalTok{CAR_TYPE, }\DataTypeTok{order =} \OtherTok{TRUE}\NormalTok{, }\DataTypeTok{levels =} \KeywordTok{c}\NormalTok{(}\StringTok{"Minivan"}\NormalTok{, }\StringTok{"z_SUV"}\NormalTok{, }\StringTok{"Van"}\NormalTok{, }\StringTok{"Pickup"}\NormalTok{, }\StringTok{"Panel Truck"}\NormalTok{, }\StringTok{'Sports Car'}\NormalTok{)))}
\NormalTok{testing}\OperatorTok{$}\NormalTok{RED_CAR =}\StringTok{ }\KeywordTok{ifelse}\NormalTok{(testing}\OperatorTok{$}\NormalTok{RED_CAR }\OperatorTok{==}\StringTok{ "no"}\NormalTok{, }\DecValTok{0}\NormalTok{, }\DecValTok{1}\NormalTok{)}
\NormalTok{testing}\OperatorTok{$}\NormalTok{RED_CAR  =}\StringTok{ }\KeywordTok{as.numeric}\NormalTok{(testing}\OperatorTok{$}\NormalTok{RED_CAR)}
\NormalTok{testing}\OperatorTok{$}\NormalTok{REVOKED =}\StringTok{ }\KeywordTok{ifelse}\NormalTok{(testing}\OperatorTok{$}\NormalTok{REVOKED }\OperatorTok{==}\StringTok{ "No"}\NormalTok{, }\DecValTok{0}\NormalTok{, }\DecValTok{1}\NormalTok{)}
\NormalTok{testing}\OperatorTok{$}\NormalTok{REVOKED  =}\StringTok{ }\KeywordTok{as.numeric}\NormalTok{(testing}\OperatorTok{$}\NormalTok{REVOKED)}
\NormalTok{testing}\OperatorTok{$}\NormalTok{URBANICITY =}\StringTok{ }\KeywordTok{ifelse}\NormalTok{(testing}\OperatorTok{$}\NormalTok{URBANICITY }\OperatorTok{==}\StringTok{ "z_Highly Rural/ Rural"}\NormalTok{, }\DecValTok{0}\NormalTok{, }\DecValTok{1}\NormalTok{)}
\NormalTok{testing}\OperatorTok{$}\NormalTok{URBANICITY  =}\StringTok{ }\KeywordTok{as.numeric}\NormalTok{(testing}\OperatorTok{$}\NormalTok{URBANICITY)}

\NormalTok{testing =}\StringTok{ }\KeywordTok{apply}\NormalTok{(testing[], }\DecValTok{2}\NormalTok{, }\ControlFlowTok{function}\NormalTok{(x) }\KeywordTok{ConvertQuatitative}\NormalTok{(x)) }\OperatorTok\KeywordTok{data.frame}\NormalTok{()}
\NormalTok{comp.data <-}\StringTok{ }\KeywordTok{mice}\NormalTok{(testing,}\DataTypeTok{m=}\DecValTok{2}\NormalTok{,}\DataTypeTok{maxit=}\DecValTok{10}\NormalTok{,}\DataTypeTok{meth=}\StringTok{'pmm'}\NormalTok{,}\DataTypeTok{seed=}\DecValTok{500}\NormalTok{)}
\end{Highlighting}
\end{Shaded}

\begin{verbatim}
## 
##  iter imp variable
##   1   1  AGE  YOJ  INCOME  HOME_VAL  JOB  CAR_AGE
##   1   2  AGE  YOJ  INCOME  HOME_VAL  JOB  CAR_AGE
##   2   1  AGE  YOJ  INCOME  HOME_VAL  JOB  CAR_AGE
##   2   2  AGE  YOJ  INCOME  HOME_VAL  JOB  CAR_AGE
##   3   1  AGE  YOJ  INCOME  HOME_VAL  JOB  CAR_AGE
##   3   2  AGE  YOJ  INCOME  HOME_VAL  JOB  CAR_AGE
##   4   1  AGE  YOJ  INCOME  HOME_VAL  JOB  CAR_AGE
##   4   2  AGE  YOJ  INCOME  HOME_VAL  JOB  CAR_AGE
##   5   1  AGE  YOJ  INCOME  HOME_VAL  JOB  CAR_AGE
##   5   2  AGE  YOJ  INCOME  HOME_VAL  JOB  CAR_AGE
##   6   1  AGE  YOJ  INCOME  HOME_VAL  JOB  CAR_AGE
##   6   2  AGE  YOJ  INCOME  HOME_VAL  JOB  CAR_AGE
##   7   1  AGE  YOJ  INCOME  HOME_VAL  JOB  CAR_AGE
##   7   2  AGE  YOJ  INCOME  HOME_VAL  JOB  CAR_AGE
##   8   1  AGE  YOJ  INCOME  HOME_VAL  JOB  CAR_AGE
##   8   2  AGE  YOJ  INCOME  HOME_VAL  JOB  CAR_AGE
##   9   1  AGE  YOJ  INCOME  HOME_VAL  JOB  CAR_AGE
##   9   2  AGE  YOJ  INCOME  HOME_VAL  JOB  CAR_AGE
##   10   1  AGE  YOJ  INCOME  HOME_VAL  JOB  CAR_AGE
##   10   2  AGE  YOJ  INCOME  HOME_VAL  JOB  CAR_AGE
\end{verbatim}

\begin{verbatim}
## Warning: Number of logged events: 2
\end{verbatim}

\begin{Shaded}
\begin{Highlighting}[]
\NormalTok{testing =}\StringTok{ }\KeywordTok{complete}\NormalTok{(comp.data)}
\NormalTok{testing =}\StringTok{ }\NormalTok{testing[, }\OperatorTok{!}\KeywordTok{names}\NormalTok{(testing) }\OperatorTok\StringTok{ }\KeywordTok{c}\NormalTok{(}\StringTok{'Index'}\NormalTok{, }\StringTok{'TARGET_FLAG'}\NormalTok{, }\StringTok{'TARGET_AMT'}\NormalTok{)]}

\NormalTok{testing}\OperatorTok{$}\NormalTok{JOB_EDU =}\StringTok{  }\KeywordTok{round}\NormalTok{(}\KeywordTok{log}\NormalTok{(testing}\OperatorTok{$}\NormalTok{JOB }\OperatorTok{*}\StringTok{ }\NormalTok{testing}\OperatorTok{$}\NormalTok{EDUCATION))}
\NormalTok{testing}\OperatorTok{$}\NormalTok{HOME_INCOME =}\StringTok{ }\KeywordTok{log}\NormalTok{(}\DecValTok{1}\OperatorTok{+}\StringTok{ }\NormalTok{testing}\OperatorTok{$}\NormalTok{HOME_VAL) }\OperatorTok{*}\StringTok{ }\KeywordTok{log}\NormalTok{(}\DecValTok{1} \OperatorTok{+}\StringTok{ }\NormalTok{testing}\OperatorTok{$}\NormalTok{INCOME)}
\NormalTok{testing}\OperatorTok{$}\NormalTok{MVR_PTS_Trans =}\StringTok{ }\KeywordTok{round}\NormalTok{(}\KeywordTok{log}\NormalTok{(}\DecValTok{1}\OperatorTok{+}\StringTok{ }\NormalTok{testing}\OperatorTok{$}\NormalTok{MVR_PTS))}
\NormalTok{testing}\OperatorTok{$}\NormalTok{AGE_SEX =}\KeywordTok{log}\NormalTok{(}\DecValTok{1} \OperatorTok{+}\StringTok{ }\NormalTok{testing}\OperatorTok{$}\NormalTok{AGE) }\OperatorTok{*}\StringTok{ }\NormalTok{(}\DecValTok{1}\OperatorTok{+}\NormalTok{testing}\OperatorTok{$}\NormalTok{SEX) }
\NormalTok{testing}\OperatorTok{$}\NormalTok{BLUEBOOK_TRAN =}\StringTok{ }\NormalTok{(testing}\OperatorTok{$}\NormalTok{BLUEBOOK}\OperatorTok{^}\NormalTok{lambda }\DecValTok{-1}\NormalTok{)}\OperatorTok{/}\NormalTok{lambda}
\end{Highlighting}
\end{Shaded}


\end{document}
